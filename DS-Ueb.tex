\documentclass[ngerman,a4paper]{report}
\usepackage[ngerman]{babel}
\usepackage[T1]{fontenc}
\usepackage[utf8]{inputenc}
\usepackage{MyriadPro}
\usepackage{geometry}
\geometry{verbose,tmargin=3cm,bmargin=3cm,lmargin=3cm,rmargin=3cm}
\usepackage{listings}
\usepackage{stmaryrd}
\usepackage{paralist}
\usepackage{array}
\usepackage{color}
\usepackage{caption}
\usepackage{url}
\definecolor{dkgreen}{rgb}{0,0.6,0}
\definecolor{gray}{rgb}{0.5,0.5,0.5}
\definecolor{mauve}{rgb}{0.58,0,0.82}

\lstdefinelanguage
[x64]{Assembler}     % add a "`x64"' dialect of Assembler
[x86masm]{Assembler} % based on the "`x86masm"' dialect
% with these extra keywords:
{morekeywords={CDQE,CQO,CMPSQ,CMPXCHG16B,JRCXZ,LODSQ,MOVSXD, %
               POPFQ,PUSHFQ,SCASQ,STOSQ,IRETQ,RDTSCP,SWAPGS, %
               rax,rdx,rcx,rbx,rsi,rdi,rsp,rbp, %
               r8,r8d,r8w,r8b,r9,r9d,r9w,r9b}} % etc.

\lstset{language=[x64]Assembler,
title=\lstname,
numbers=left,
numberstyle=\tiny\color{gray},
stepnumber=1,
numbersep=5pt,
%frame = single,
tabsize =2,
breaklines = true,
breakatwhitespace = false,
keywordstyle=\color{blue},          % keyword style
commentstyle=\color{dkgreen},       % comment style
stringstyle=\color{mauve},         % string literal style
literate=%
{Ö}{{\"O}}1
{Ä}{{\"A}}1
{Ü}{{\"U}}1
{ß}{{\ss}}2
{ü}{{\"u}}1
{ä}{{\"a}}1
{ö}{{\"o}}1
}
\providecommand{\tabularnewline}{\\}

\usepackage{fancyhdr}
\pagestyle{fancy}
\usepackage{lastpage}
\makeatletter

\lhead{\textbf{\@title} \\ \@author}
\chead{}
\rhead{\@date \\ \thepage \ von \pageref{LastPage}}
\cfoot{}

\renewcommand{\maketitle}{}

\renewcommand{\familydefault}{\sfdefault}

 
\author{Hinnerk van Bruinehsen}
\title{Distributed Systems\\Übungen}
\date{SoSe 2013}

\begin{document}
\maketitle
\chapter{Ueb1}
\section*{Assignment 1. Client-Server}
The main problem is that the time a client waits for a response from a server increases since each Server waits for a response from the next server itself.\\
This isn't inherently a problem of the client-server model but concerns every architecture that works over multiple nodes (chain ... ).
i

\section*{Assignment 2. Unstructured Overlay}
1. Wahrscheinlichkeit: $P = \frac{c-1}{n-2}+\frac{c-1}{n-2}$: R ist ein Nachbar, daher $c-1$, R und der knoten selbst können keine Nachbarn mehr sein, also $n-2$. Wir haben 2 Fälle (P mit Q verbunden und Q mit P verbunden)

2. Wenn L Anzahl der erreichbaren Knoten ist, $C \leq L \leq C^2$  
\section*{Assignment 3. Structured Overlay}
Wenn nach der Topologie des Overlaynetworks geroutet wird, kann es sein, dass die Verbindungen im darüber liegenden Netz sehr ungünstig sind und es eigentlich bessere Verbindungen gäbe.

\end{document}
