\chapter{Leader Election algorithms}
\section{leader election in a synchronnous ring}
Network is a graph $G$ consisting of $n$ nodes connected by unidirectional links. Use $\mod n$ for labels\\
\begin{compactitem}
\item elected node is ``leader''
\item leader election is not possible for identical processes/nodes
\end{compactitem}

\subsection{LCR algorithm}
(Lelan, Chang, Roberts)\\

\begin{compactitem}
\item unidirectional communication
\item ring size unknown
\item only leader produces output
\item algorithm compares UID
\end{compactitem}

\begin{lstlisting}
For each node
	a = a UID, initially i's UID
	send = a UID or NULL, initially i's UID
	status = {unknown, leader} initially unknown

message generation
	send = current value of send to node i+1

state transitions
	send = NULL
	if incoming message is v (a UID) then
		v>u: 	send v
		v=u: 	status=leader
		v<u: 	do nothing
\end{lstlisting}

\textbf{Correctness}\\
Let max index of process with $max(UID)$ let $u_{max}$ is its UID\\
Show:\\
\begin{compactitem}
\item[(i)] process max outputs ``leader'' after $n$ rounds
\item[(ii)] no other process does the same
\end{compactitem}

We clarify:\\
\begin{compactitem}
\item[(iii)] After $n$ rounds status$_{max}$=leader
\end{compactitem}
and\\
\begin{compactitem}
\item[(iv)] For $0\leq r\leq n-1$ after $r$ rounds\\
	$send_{max}=u_{max}$
\end{compactitem}
find UID at distance $r$ from $i_{max}$ as it has t og once around.\\

Show $(iv)$ for all r: Induction\\
then (iii)\\

\textbf{Complexity}\\
\begin{compactitem}
\item time complexity id $n$ rounds
\item communocation complexity $\mathcal{O}(n^2)$
\item not very expensive in time many messages

\end{compactitem}

\subsection{Algorithm of Hirschberg and Sinclair (HS-Alg)}
- reduces number of messages to $\mathcal{O}(n \log n)$\\

\begin{lstlisting}
each process has states with components
	u, UID: initially i's UID
	send+ containing NULL or (UID, flag{in, out}, hopcount): initially (i's UID, out, 1)
	send- as send+
	status $\in$E{unknown, leader} initailly unknown phase $\in \mathbb{N}$: initially 0

message generation
	send current send+ to process i+1
	send current send- to process i-1

state transitions
	send+=NULL
	send-=NULL
	if message from (i-1) is (v, out, h) then
		v>u $\land$ h>1: send+ = (v,out,h-1)
		v>u $\land$ h=1: send- = (v,in,1)
		v=u status = leader
	if message from i+1 is (v, out, h) then
		v>u $\land$ h>1: send- = (v,out, h-1)
		v>u $\land$ h=1: send+ = (v,in,1)
		v=u status=leader
	if message from i-1 is (v,in,1) $\land v\neq u$ then
		send+=(v,in,1)
	if message from i+1 is (v,in,1) $\land v\neq u$ then
		send-=(v,in,1)
	if both messages from i-1 and i+1 are (u,in,1) then
		phase++
		send+=(u,out, $2^{phase}$)
		send-=(u,out, $2^{phase}$)
\end{lstlisting}
\textbf{Complexity}\\
Total number of phases is at most $1+\lceil\log(n)\rceil$\\
the total number of messages is at mostin $(1+\lceil(\log(n))\rceil\ \approx \mathcal{O}(n\log n)$\\
Total time complexity is at most $3n$ if $n$ power of $2$ other wise is $5n$\\

\subsection{Time slice algorithm}
\begin{compactitem}
\item ring size $n$ is known
\item unidirectional
\item elects minimum
\end{compactitem}*


\begin{lstlisting}
	phases with n rounds
	in phase r consisting of rounds (v-1)n+1,\dots,vn
	only a token carrying UID v is permitted
	if a process  with UID v exists, thenit elects itself leader and sends a token wit it's UID

\end{lstlisting}
\textbf{Complexity}: number of messages is n, time complexity $n\cdot u_{min}$

\subsection{Variable speeds algorithm}
\begin{compactitem}
\item each process $i$ creates a token to tracel around the ring, carrying UID u of origin
\item tokens travel at diffeneed speed
\item token carrying UID v travels  1 messages every $2^{v}$ rounds
\item each process memorises smallestUID
\item return to origin elects UID
\end{compactitem}

\textbf{Complexity}\\
\dots


How many messages in total? $\sum\limits_{k=1}^n \frac{1}{2^{k-1}} (<2n)$\\
Time complexity: $n\cdot 2^{u_{min}}$

\section{Leader election in a wireless environment}
\begin{compactitem}
\item consider time needed for communication
\item nodes sest up a tree
\item select based on information like battery lifetime
\item node issues leader request to all it's neighbours
\item becomes parent if there is none yet
\end{compactitem}
\begin{figure}[h]
	\centering
	\includegraphics[width=400px]{gfx/Leader_adhoc.png}
	\caption{Election in wireless networks}
	\label{img:Leader_adhoc}
\end{figure}

\section{The Bully Algorithm(flooding) (Garcia-Mdina, 1982)}
- process P holds election
\begin{compactenum}
\item P sends ELECT message to all processes
\item P wins if there is no response $\Rightarrow$ P is leader and sends COORDINATOR message to all available nodes.
\item if Q answers with OK, Q takes over and sends ELECT again.
\end{compactenum}
\begin{figure}[h]
	\centering
	\includegraphics[width=400px]{gfx/leader1.png}
	\caption{The bully election algorithm}
	\label{img:leader1}
\end{figure}

%BILD DINGDINGDING

