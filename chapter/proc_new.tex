Other focus than in OS

\section{Threads}
Much more lightweight than processes
only the bare minimum of management data (CPU context (registers) and thread management data)
Safety is to be done by the designer/programmer

Threads are often provided as Threadpackages 
two approaches possible: 
- user based (cheap creation and destruction but blocking syscall blocks whole process)
- kernel threads (expensive due to context switches)
- LWP (light weight processes, hybrid mode, creation and destroctino expensive but seödpom necessarys)

Benefits for Distributed systems: blocking syscall blocks thread, not the entire pro cess

Multithreaded Clients (webbrowsers, possible benefit: usage of several servers simultaneusly)

Multithreaded Servers (one dispatcher, several workers, enables working even when a thread is blocked)

Virtualization

enabels hardware independence
2 kinds: process virt: for a single process (e.g. wine), full system virt. :VMM (Virtual machine monitor (hypervisor, like vmware, xen...)

Network user interfaces (e.g. X windows) 

Thin clients

Server 
iterative: wait - serve -wait
concurrent: dispatch request to other process/thread and immediatly wait for next request


normal server vs. superserver
superserver starts servers according to clientrequests (e.g. inetd)

stateful vs stateless (vs. soft state)

server cluster
often organised in three tiers (Logical switch, application/compute servers, distributed file/database system)

Managing Clusters:
different approaches, e.g. every node individual or via an administration interface that replicates commands to different nodes


distributed servers

managing server clusters

Code migration
Move a process between machines (e.g. for load balancing)
