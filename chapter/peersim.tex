\chapter{Peersim}
\section{Kinds of simulation}
\textbf{cycle-based}\\
\begin{compactitem}
    \item fixed cycles; every cycle the part in ``next\_cycle'' is run
    \item \textbf{Observer:} after every cycle
\end{compactitem}

\textbf{event-based}\\
\begin{compactitem}
    \item exchange of messages, the part in ``processevent'' is run (e.g. message delays)
    \item \textbf{Observer:} at specified point of time
\end{compactitem}

\section{Configuration}
Specified are controls:\\
\begin{compactitem}
    \item protocols
    \item \textbf{Initializer:} run at the beginning of the simulation
    \item \textbf{Observer:} run during simulation, change nothing, at least one needed
\end{compactitem}

experiments (number of repetitions), range (simulation is repeated with different parameters)
\section{Initializer}
\begin{compactitem}
    \item \textbf{wirekout} parameter k (number of nodes, normally random), creates a ``wireplan'' for the simulation
    \item \textbf{peekdistributioninitializer} creates one maximum and sets everything else to zero
    \item \textbf{lineardistributioninitializer} creates linear distribution from start to end value
\end{compactitem}

\section{Stats, order of parameters}
\texttt{obs\_name time min max n avg variance ct\_min ct\_max}

\section{PeerSim simulation life-cycle}

PeerSim was designed to encourage modular programming based on objects (building blocks). Every block is easily replaceable by another component implementing the same interface (i.e., the same functionality). The general idea of the simulation model is:\\

\begin{compactenum}
\item choose a network size (number of nodes)
\item choose one or more protocols to experiment with and initialize them
\item choose one or more Control objects to monitor the properties you are interested in and to modify some parameters during the simulation (e.g., the size of the network, the internal state of the protocols, etc)
\item run your simulation invoking the Simulator class with a configuration file, that contains the above information
\end{compactenum}

\section{Interfaces}
Node, CDProtocol, Linkable, Control

\section{Example: Average}

\subsection{AverageFunction}

\begin{lstlisting}[language=java, basicstyle=\footnotesize]
package example.aggregation;

import peersim.cdsim.CDProtocol;
import peersim.config.FastConfig;
import peersim.core.CommonState;
import peersim.core.Linkable;
import peersim.core.Node;
import peersim.vector.SingleValueHolder;

/**
 * This class provides an implementation for the averaging function in the
 * aggregation framework. When a pair of nodes interact, their values are
 * averaged. The class subclasses {@link SingleValueHolder} in order to provide
 * a consistent access to the averaging variable value.
 * 
 * Note that this class does not override the clone method, because it does not
 * have any state other than what is inherited from {@link SingleValueHolder}.
 */
public class AverageFunction extends SingleValueHolder implements CDProtocol {

    /**
     * Creates a new {@link example.aggregation.AverageFunction} protocol
     * instance.
     * 
     * @param prefix
     *            the component prefix declared in the configuration file.
     */
    public AverageFunction(String prefix) {
        super(prefix);
    }

    /**
     * Using an underlying {@link Linkable} protocol choses a neighbor and
     * performs a variance reduction step.
     * 
     * @param node
     *            the node on which this component is run.
     * @param protocolID
     *            the id of this protocol in the protocol array.
     */
    public void nextCycle(Node node, int protocolID) {
        int linkableID = FastConfig.getLinkable(protocolID);
        Linkable linkable = (Linkable) node.getProtocol(linkableID);
        if (linkable.degree() > 0) {
            Node peer = linkable.getNeighbor(CommonState.r.nextInt(linkable
                    .degree()));

            // Failure handling
            if (!peer.isUp())
                return;

            AverageFunction neighbor = (AverageFunction) peer
                    .getProtocol(protocolID);
            double mean = (this.value + neighbor.value) / 2;
            this.value = mean;
            neighbor.value = mean;
        }
    }
}
\end{lstlisting}

\subsection{AverageObserver}

\begin{lstlisting}[language=java, basicstyle=\footnotesize]
package example.aggregation;

import peersim.config.*;
import peersim.core.*;
import peersim.vector.*;
import peersim.util.IncrementalStats;

/**
 * Print statistics for an average aggregation computation. Statistics printed
 * are defined by {@link IncrementalStats#toString}
 */
public class AverageObserver implements Control {
    /**
     * Config parameter that determines the accuracy for standard deviation
     * before stopping the simulation. If not defined, a negative value is used
     * which makes sure the observer does not stop the simulation
     * 
     * @config
     */
    private static final String PAR_ACCURACY = "accuracy";

    /**
     * The protocol to operate on.
     * 
     * @config
     */
    private static final String PAR_PROT = "protocol";

    /**
     * The name of this observer in the configuration. Initialized by the
     * constructor parameter.
     */
    private final String name;

    /**
     * Accuracy for standard deviation used to stop the simulation; obtained
     * from config property {@link #PAR_ACCURACY}.
     */
    private final double accuracy;

    /** Protocol identifier; obtained from config property {@link #PAR_PROT}. */
    private final int pid;

    /**
     * Creates a new observer reading configuration parameters.
     */
    public AverageObserver(String name) {
        this.name = name;
        accuracy = Configuration.getDouble(name + "." + PAR_ACCURACY, -1);
        pid = Configuration.getPid(name + "." + PAR_PROT);
    }

    /**
     * Print statistics for an average aggregation computation. Statistics
     * printed are defined by {@link IncrementalStats#toString}. The current
     * timestamp is also printed as a first field.
     * 
     * @return if the standard deviation is less than the given
     *         {@value #PAR_ACCURACY}.
     */
    public boolean execute() {
        long time = peersim.core.CommonState.getTime();

        IncrementalStats is = new IncrementalStats();

        for (int i = 0; i < Network.size(); i++) {

            SingleValue protocol = (SingleValue) Network.get(i)
                    .getProtocol(pid);
            is.add(protocol.getValue());
        }

        /* Printing statistics */
        System.out.println(name + ": " + time + " " + is);

        /* Terminate if accuracy target is reached */
        return (is.getStD() <= accuracy);
    }
}
\end{lstlisting}
