\chapter{Peersim}
\section{Kinds of simulation}
\textbf{cycledriven}\\
\begin{compactitem}
    \item fixed cycles; every cycle the part in ``next\_cycle'' is run
    \item \textbf{Observer:} after every cycle
\end{compactitem}

\textbf{eventdriven}\\
\begin{compactitem}
    \item exchange of messages, the part in ``processevent'' is run (e.g. message delays)
    \item \textbf{Observer:} at specified point of time
\end{compactitem}

\section{Configuration}
Specified are controls:\\
\begin{compactitem}
    \item protocols
    \item \textbf{Initializer:} run at the beginning of the simulation
    \item \textbf{Observer:} run during simulation, change nothing, at least one needed
\end{compactitem}

experiments (number of repetitions), range (simulation is repeated with different parameters)
\section{Initializer}
\begin{compactitem}
    \item \textbf{wirekout} parameter k (number of nodes, normally random), creates a ``wireplan'' for the simulation
    \item \textbf{peekdistributioninitializer} creates one maximum and sets everything else to zero
    \item \textbf{lineardistributioninitializer} creates linear distribution from start to end value
\end{compactitem}

\section{Stats, order of parameters}
\texttt{obs\_name time min max n avg variance ct\_min ct\_max}
