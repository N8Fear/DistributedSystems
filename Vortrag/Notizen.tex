\documentclass[10pt]{report}
\usepackage[english,ngerman]{babel}
\usepackage{times}
\usepackage[utf8]{inputenc}
\usepackage{geometry}
\geometry{a5paper}

\begin{document}
\section{Überblick}
\begin{itemize}
	\item \textbf{probabilitischer Informationsaustausch\\}
	keine Garantien
\item \textbf{Wiederholung der einzelnen Arbeitsschritte (endlos)\\}


\item \textbf{analog zur Gerüchteverbreitung oder zu Krankheitsepidemien\\}
	disease - contaminated - infect - epidemics
\item \textbf{historisch zur Sicherung der Konsistenz verteiler Datenbanken\\}
	inzwischen viele Anwendungsgebiete (später mehr)\\
	über 20 Jahre alt, aber erst in den letzen Jahren beliebter geworden.\\
	robust, indifferent gegenüber veränderungen in der gruppenzusammensetzung
 	in verteilten systemen benötigt	wg. continous change $\rightarrow$ convergent behaviour\\
\end{itemize}
\section{Struktur}
Nach ``Gossiping in Distributed Systems'', Anne-Marie Kermarrec, Maarten van Steen\\
Einschub:\\
\textbf{Begriffserklärung Peers:}\\
\begin{itemize}
	\item \textbf{Prozesse\\}
		possibly dyn changing
	\item \textbf{haben Cache  mit Referenzen zu anderen Peers}
	\item \textbf{ggf. auch peer-spezifische Informationen im cache\\}
		z.B. gewichtung\\

\end{itemize}
Peer 1 wählt zufälligen Peer 2 für Kommunikation $\rightarrow$ wählt Information und sendet an 2 $\rightarrow$ 2 wählt Information und sendet an 1 $\rightarrow$ evtl. müssen Daten aus dem Cache ersetzt werden, wenn dieser voll ist.
\subsection{Peerauswahl}
\begin{itemize}
	\item \textbf{verschiedene Auswahlkriterien je nach Anwendung\\}
	z.B. - Zufall, Auswahl nach bestimmten Kriterien
\item \textbf{Unterschiede bei Auswahl über kabellose oder kabelgebundene Verbindungen\\}
	kabellos kann nur peers in reichweite wählen oder verliert Datendurchsatz falls peers als ``Proxy'' arbeiten müssen
\item \textbf{Simulation eines anderen Verbindungstyps möglich, häufig teuer und unnötig}
\item \textbf{kaum Unterschiede auf Applikationsschicht zwischen synchron und asynchron\\}
	asynchron heißt, dass peer 2 keine infos an peer 1 zurückgibt
\item \textbf{asynchron ist kein "richtiges" Gossiping}
\end{itemize}
\subsection{Datenaustausch}
\begin{itemize}
	\item \textbf{peers entscheiden, welche Daten sie austauschen}
	\item \textbf{entweder Applikationsdaten oder Referenzen zu anderen peers werden}
  ausgetauscht
\end{itemize}
\section{Datenverarbeitung}
\begin{itemize}
	\item \textbf{stark anwendungsabhängig}
\end{itemize}
\end{document}
